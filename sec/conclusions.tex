In the analysis we show that \ml techniques are indeed able to make accurate predictions on lumps and the \wzw model in \sft.
In fact decision trees and \emph{ANN} models showed promising results on a model dependent basis (i.e.\ with knowledge of the underlying physics model).
In principle it seems to also be possible to merge different datasets and produce meaningful predictions using \emph{ANN} models (and \emph{r-SVR} in some cases) which displayed the best adaptivity to different datasets.

It is still not clear whether it is possible to use algorithms trained on different datasets (even coming from diverse physical models) and make accurate predictions on different data, produced by a different physical model.
However when introducing the double lumps directly in the evaluation and test sets, the \emph{ANN}s showed ability to generalise to these solutions.
As shown by the coefficient of determination \rr, the variance of the data is greatly explained by the model, which can therefore be used for predictions.
